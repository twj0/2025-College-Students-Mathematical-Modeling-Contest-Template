% CUMCM 2025 Paper Template
% For electronic submission, use the 'withoutpreface' option.
% Official requirement: https://cumcm.cnki.net/cumcm/studentHome/noticeDetail?id=c0c5220f-0226-4cd8-bce1-e1f81e1e23f6

\documentclass[withoutpreface,bwprint]{cumcm} % Use 'cumcm.cls' from the same directory

% Basic packages (already included in cumcm.cls but good to be aware of)
\usepackage{url}
\usepackage{graphicx}
\usepackage{amsmath}
\usepackage{booktabs}
\usepackage{subcaption}

% Title and Team Information (To be filled in the solution/main.tex)
\title{全国大学生数学建模竞赛 \LaTeX{} 模板}
\tihao{B}
\baominghao{12345}
\schoolname{Your University}
\membera{Member A}
\memberb{Member B}
\memberc{Member C}
\supervisor{Supervisor}
\yearinput{2025}
\monthinput{9}
\dayinput{1}

\begin{document}

\maketitle

\begin{abstract}
%% TODO: Write the abstract here. It should be a concise summary of your work (around 300 words).
%% It must include the problem description, your model, the algorithm, and the main conclusions.
%% AI-PROMPT: Based on the entire paper, please generate a concise and comprehensive abstract.

\keywords{Keyword 1\quad Keyword 2\quad Keyword 3} %% TODO: Fill in your keywords
\end{abstract}

% The table of contents is not required by CUMCM.
% \tableofcontents

\section{问题重述 (Problem Restatement)}
%% TODO: Clearly and concisely restate the problem in your own words.
%% This section demonstrates your understanding of the problem.

\section{模型假设 (Model Assumptions)}
%% TODO: List all the assumptions you made for your model.
%% Justify why these assumptions are reasonable.
\begin{assumption}
    Assumption 1...
\end{assumption}
\begin{assumption}
    Assumption 2...
\end{assumption}

\section{符号说明 (Notation)}
%% TODO: Define all the symbols and variables used in your paper.
\begin{center}
    \begin{tabular}{cl}
        \toprule[1.5pt]
        \makebox[0.3\textwidth][c]{符号 (Symbol)} & \makebox[0.5\textwidth][c]{意义 (Meaning)} \\
        \midrule[1pt]
        $x$ & Description of variable x. \\
        $Y$ & Description of variable Y. \\
        \bottomrule[1.5pt]
    \end{tabular}
\end{center}

\section{模型建立与求解 (Model Formulation and Solution)}
\subsection{模型一:... (Model 1: ...)}
%% AI-PROMPT: Based on the code in `src/solution/problem1.py`, explain the mathematical principles
%% of the model, including the equations and the logic behind them.

\subsection{模型求解 (Model Solution)}
%% TODO: Describe the algorithm or method used to solve the model.
%% Refer to the implementation in the source code.

\section{结果分析 (Results Analysis)}
%% TODO: Present and analyze your results. Use figures and tables to support your analysis.
\begin{figure}[htbp]
    \centering
    % \includegraphics[width=0.8\textwidth]{../solution/figures/figure_name.png}
    \caption{Caption for the figure.}
    \label{fig:example_figure}
\end{figure}
%% AI-PROMPT: Please describe the information revealed by the figure above (Fig \ref{fig:example_figure})
%% and analyze its impact on our model selection and conclusions.

\begin{table}[htbp]
    \centering
    \caption{Caption for the table.}
    \label{tab:example_table}
    \begin{tabular}{ccc}
        \toprule
        Header 1 & Header 2 & Header 3 \\
        \midrule
        Data 1 & Data 2 & Data 3 \\
        \bottomrule
    \end{tabular}
\end{table}
%% AI-PROMPT: Based on the data in `results/tables/table_name.csv`, summarize the key findings
%% from the table above (Table \ref{tab:example_table}).

\section{结论与展望 (Conclusion and Outlook)}
%% TODO: Summarize your findings and the main conclusions of your work.
%% Discuss the strengths and weaknesses of your model and suggest potential improvements.

% References
\begin{thebibliography}{9}
    \bibitem{bib:example}
    Author, A. N.
    \newblock Title of the work.
    \newblock \emph{Journal Name}, Year, Volume(Issue), pages.
\end{thebibliography}

% Appendices
\begin{appendices}
\section{Appendix A: Source Code}
%% TODO: Include key parts of your source code if necessary.
%% Use the lstlisting environment for code snippets.
\end{appendices}

\end{document}