% CUMCM 2025 Solution Paper
% This is the main file for your competition paper.

% We use the class file from the template directory.
\documentclass[withoutpreface,bwprint]{../template/cumcm}

\usepackage{graphicx} % Required for including images
\usepackage{booktabs} % For professional quality tables
\usepackage{amsmath}  % For advanced math typesetting

% Set the graphic path to include the results and local figures directory
\graphicspath{{../../results/figures/}{figures/}}

% --- Fill in Your Team's Information ---
\title{A Solution to Problem B of CUMCM 2024} % TODO: Change to your title
\tihao{B}
\baominghao{YOUR_TEAM_ID} % TODO: Change to your team ID
\schoolname{Your University}
\membera{Student A}
\memberb{Student B}
\memberc{Student C}
\supervisor{Your Supervisor}
\yearinput{2025}
\monthinput{9}
\dayinput{1}
% -----------------------------------------

\begin{document}

\maketitle

\begin{abstract}
%% TODO: Write your abstract here.
%% AI-PROMPT: Based on the entire paper, please generate a concise and comprehensive abstract.

\keywords{Keyword 1\quad Keyword 2\quad Keyword 3} %% TODO: Fill in your keywords
\end{abstract}

\section{问题重述 (Problem Restatement)}
%% TODO: Restate the problem.

\section{模型假设 (Model Assumptions)}
%% TODO: List your model assumptions.

\section{符号说明 (Notation)}
%% TODO: Define your notation.

\section{模型建立与求解 (Model Formulation and Solution)}
%% TODO: Detail your model and how you solved it.

\section{结果分析 (Results Analysis)}
%% TODO: Analyze your results.
\begin{figure}[htbp]
    \centering
    % Example: \includegraphics[width=0.8\textwidth]{figure1_data_distribution.png}
    \caption{Your figure caption.}
    \label{fig:your_figure}
\end{figure}

\section{结论与展望 (Conclusion and Outlook)}
%% TODO: Conclude your paper.

% Use an external .bib file for references for better management
\bibliography{../ref/references}
\bibliographystyle{plain} % Or any other style you prefer

\begin{appendices}
\section{Appendix A: Code}
%% TODO: Add important code snippets if necessary.
\end{appendices}

\end{document}